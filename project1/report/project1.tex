\documentclass[acmconf,nonacm=true]{acmart}
\authorsaddresses{}

\usepackage{multirow}
\usepackage{enumitem}
\usepackage{graphicx}
\usepackage{subcaption}
\usepackage{amsmath}

\usepackage[utf8]{inputenc}
\usepackage{a4wide}
% \usepackage{amsmath}
\usepackage{verbatim}
\usepackage{array}
\usepackage{clrscode3e}

\usepackage{multicol}

\begin{document}
\title{INFO0902 - Data Structures and Algorithms \\ 
Project 1 - Selection Algorithms}

% Enter your names here, along with your student IDs.
\author{Duy Vu Dinh (\texttt{s2401627})}

\maketitle

%-------------------------------------------------------------------



% \section*{Overview}

This report presents a theoretical and experimental analysis of four selection algorithms: \texttt{SelectionSelect}, \texttt{HeapSelect}, \texttt{QuickSelect}, and \texttt{FRSelect}. The objective of each algorithm is to determine the $k$-th smallest element in an array.


\section{Theoretical analysis}

\subsection{Complexity table and justifications} \label{sec:theory-complexity-results}

Table \ref{tab:complexity} shows the time and space complexities in the worst case and the best case of the implemented algorithms, corresponding to their pseudocodes provided in Appendix \ref{Appendix}.

    \begin{table}[h]
        \centering
        \footnotesize
        \caption{Complexity of selection algorithms}
        \label{tab:complexity}
        \begin{tabular}{lcccc}
            \hline
            Algorithm & \multicolumn{2}{c}{Time complexity} & \multicolumn{2}{c}{Space complexity} \\
            \cmidrule(r){2-3} \cmidrule(l){4-5}
            & Worst case & Best case & Worst case & Best case \\
            \hline
            \textsc{SelectionSelect} & $\Theta(Nk)$ & $\Theta(Nk)$ & $\Theta(1)$ & $\Theta(1)$ \\
            \textsc{HeapSelect} & $\Theta(N + k \log N)$ & $\Theta(N + k \log N)$ & $\Theta(1)$ & $\Theta(1)$ \\
            \textsc{QuickSelect} & $\Theta(N^2)$ & $\Theta(N)$ & $\Theta(N)$ & $\Theta(\log N)$ \\
            \textsc{FRSelect} & $\Theta(N)$ & $\Theta(N)$ & $\Theta(\log N)$ & $\Theta(\log N)$ \\
            \hline
        \end{tabular}
    \end{table}

% Add citation for FRSelect https://en.wikipedia.org/wiki/Floyd%E2%80%93Rivest_algorithm or https://core.ac.uk/download/pdf/82672439.pdf

\subsubsection{\texttt{SelectionSelect}}

    \begin{itemize}
        \item \textbf{Time complexity}: The algorithm loops and compares every time. For each of the first $k$ positions, the algorithm performs up to $N$ comparisons. Therefore, the time complexity in both cases is $\Theta(Nk)$.
        \item \textbf{Space complexity}: The algorithm works in-place, meaning it runs without needing extra memory. So, $\Theta (1)$ is the complexity for both cases.
    \end{itemize}

\subsubsection{\texttt{HeapSelect}}

    \begin{itemize}
        \item \textbf{Time complexity}: A heap building stage takes $\Theta(N)$. Then, a heapify takes $\Theta(\log N)$. Therefore, the whole algorithm takes $\Theta(N + k \log N)$.
        \item \textbf{Space complexity}: The algorithm works in-place, meaning it runs without needing extra memory.
    \end{itemize}

\subsubsection{\texttt{QuickSelect}}

    \begin{itemize}
        \item \textbf{Time complexity}
        \begin{itemize}
            \item \textbf{Best case}: The pivot divides the array into two balanced parts. Since the algorithm recurses only on one side, a subarray, the expected number of comparisons follows the recurrence $T(N) = T(N/2) + \Theta(N)$, resulting in a time complexity of $\Theta(N)$. The best case usually occurs for a random array.
            \item \textbf{Worst case}: The pivot is always the smallest or largest element (e.g., for sorted arrays when the last element is used as pivot), the recursion depth becomes $N$, and the recurrence becomes $T(N) = T(N - 1) + \Theta(N)$, resulting in $\Theta(N^2)$. 
        \end{itemize}
    
        \item \textbf{Space complexity}. The space complexity depends on the depth of the recursive calls. 
        \begin{itemize}
            \item \textbf{Best case}: It is $\Theta(\log N)$ due to the recursive stack. 
            \item \textbf{Worst case}: It becomes $\Theta(N)$ if the recursion goes as deep as the array size.
        \end{itemize}
    \end{itemize}

\subsubsection{\texttt{FRSelect} (Floyd-Rivest select algorithm)}

    \begin{itemize}
        \item \textbf{Time complexity}
        \begin{itemize}
            \item The FRSelect algorithm improves QuickSelect by estimating a good pivot range using sampling and probabilistic analysis. It reduces the likelihood of unbalanced partitions. 
            \item This approach ensures a worst-case and average-case time complexity of $\Theta(N)$, mentioned in Kiwiel and Krzysztof's work \cite{kiwiel2005floyd}.
        \end{itemize}

        \item \textbf{Space complexity}
        \begin{itemize}
            \item The algorithm involves recursive pivot range refinement, which results in a space complexity of $\Theta(\log N)$ due to the recursion stack.
        \end{itemize}
    \end{itemize}

% \begin{table}[h]
%     \centering
%     \begin{tabular}{lcccc}
%         \caption{Complexity of selection algorithms}
%         \hline
%         % Algorithm & \multicolumn{2}{c|}{Space complexity} & \multicolumn{2}{c|}{Time complexity} \\
%         % \cline{2-5}
%         % & Worst case & Best case & Worst case & Best case \\
%         % \hline
%         % SelectionSelect &  &  &  &  \\
%         % HeapSelect &  &  &  &  \\
%         QuickSelect & $\Theta(n)$ & $\Theta(1)$ & $\Theta(n^2)$ & $\Theta(n)$ \\
%         FRSelect &  &  &  &  \\
%         \hline
%     \end{tabular}
    
%     \label{tab:sorting_comparison}
% \end{table}

\subsection{Complexity analysis of the average case of \texttt{Quickselect} in the case where $k=0$ (the minimum)}

\subsubsection{Intuition}
    \texttt{QuickSelect} uses the same partitioning logic as \texttt{QuickSort}. However, unlike \texttt{QuickSort} which recursively explores both sides, \texttt{QuickSelect} only recurses into one subarray, either the left or the right, depending on where the $k$-th element lies.
    
    In our case: $k = 0$ (i.e., looking for the minimum): If the pivot ends up in position 0 (smallest), we’re done. Otherwise, we only work on the left side of the pivot (elements are smaller than it).

    On average, about half of the array each time is eliminated, a partition step that costs $\Theta(N)$. And only one recursive path is gone down instead of two (unlike \texttt{QuickSort}). Therefore, the average time complexity of \texttt{QuickSelect} is $\Theta(N)$.

\subsubsection{Mathematical model}
    \begin{itemize}
        \item Number of elements in the array must be positive: $N > 0$
        \item Number of comparisons for partitioning: $N - 1$.
        \item Probability that the pivot is at position $i$: $\frac{1}{N}$.
        \item Sizes of the sub-array in this case: $i-1$
        \begin{itemize}
            \item With the assumption of $k=0$, it means looking for the smallest.
            \item Unlike \texttt{QuickSort}, which recursively explores both sides, \texttt{QuickSelect} only recurses into one subarray, the left subarray $A[0, i-1]$ when $k=0$.
        \end{itemize}
    \end{itemize}
    So, the recurrence is:

    \begin{equation}
        C_0 = 0
    \end{equation}

    \begin{equation} \label{eq:2}
        C_N = (N-1) + \sum_{i=1}^{N-1} \frac{1}{N} C_i
    \end{equation}

    Multiply Eq. (\ref{eq:2}) by $n$:
    \begin{equation} \label{eq:3}
        NC_N = N(N-1) + \sum_{i=1}^{N-1} C_i
    \end{equation}

    The same formula of Eq. (\ref{eq:4}) for $(N - 1)$:
    \begin{equation} \label{eq:4}
        (N-1)C_{N-1} = (N-1)(N-2) + \sum_{i=1}^{N-2} C_i
    \end{equation}

    Eq. (3) $-$ Eq. (4):
    \begin{align} 
        NC_N - (N - 1)C_{N-1} &= N (N - 1) - (N - 1)(N - 2) + C_{N-1} \label{eq:5} \\
        \llap{$\Leftrightarrow$ \qquad} NC_N &= NC_{N-1} + 2(N-1) \label{eq:6} \\
        \llap{$\Leftrightarrow$ \qquad} C_N &= C_{N-1} + \frac{2(N-1)}{N}  \label{eq:7}
    \end{align}

    Then, Eq. (\ref{eq:7}) is telescoped:
    \begin{align} \label{eq:8}
        C_N &= C_1 + \sum_{a=2}^{N} \frac{2(a-1)}{a} \\
        % \llap{$\Leftrightarrow$ \qquad} C_N &= C_1 + (2N - 1) - 2 \sum_{a=2}^{N} \frac{1}{a} \\
        \llap{$\Leftrightarrow$ \qquad} C_N &= C_1 + 2N - 2 \sum_{a=1}^{N} \frac{1}{a}
    \end{align}

    The sum of $\sum_{a=1}^{N} \frac{1}{a}$ simplifies into a form involving harmonic numbers $H_N = \sum_{a=1}^{N} \frac{1}{a}$, then:

    \begin{align}
        C_N &= 2N - 2 H_N  + C_1
    \end{align}

    Due to $H_N \in \Theta (\log N)$, therefore:
    
    \begin{equation}
        C_N \in \Theta(N)
    \end{equation}

So, the expected complexity to find the minimum ($k = 0$) using \text{QuickSelect} is $\Theta(N)$.

    
\subsection{The stable of algorithm}

All of the mentioned algorithms are unstable. Justifications:
\begin{itemize}
    \item \texttt{SelectionSelect} swaps non-adjacent elements.
    \item \texttt{HeapSelect} includes Heapify and extract-min disrupt the original positions of duplicates.
    \item Both of \texttt{QuickSelect} and \texttt{FRSelect} use partitioning moves to equal elements without considering original order (not order-preserving).
\end{itemize}




\newpage

\section{Experiments}

\subsection{Experiment results}

% https://docs.google.com/spreadsheets/d/1T8_v9qw1nP2TVHZsp8hWlpYfy9XIOhXbvHXGW2Oq3hg/edit?gid=1690685856#gid=1690685856

The experiment results, including the average computation times and the average number of comparisons over 10 experiments, of each mentioned algorithm for the search of median and $10$-th percentile are shown in Table \ref{table:k-0.5N} and Table \ref{table:k-0.1N}, respectively. Unfortunately, due to the limitation of resources, I cannot run the cases for $N=10^6$ for the \texttt{SelectionSelect} and \texttt{QuickSelect} algorithms. However, the experiments regarding $N=10^4$ and $N=10^5$ are quite enough for them to analyze the complexity. In addition, the computation times and the number of comparisons for each algorithm with $N=10^6$ are expected to follow the theoretical evolution of complexity in Section \ref{sec:comment-a}.

\begin{table}[h!]
    \centering
    \scriptsize
    \caption{Computation times and the number of computations of the algorithms for the search of the median ($k=N/2$).}
    \label{table:k-0.5N}
    \begin{tabular}{llrrrrrrrrrrrr}
        \toprule
        {Metric} & {Type of array} & \multicolumn{3}{c}{{random}} & \multicolumn{3}{c}{{increasing}} & \multicolumn{3}{c}{{decreasing}} & \multicolumn{3}{c}{{constant}} \\
        \cmidrule(r){3-5} \cmidrule(lr){6-8} \cmidrule(lr){9-11} \cmidrule(l){12-14}
        & {Size} & \multicolumn{1}{c}{$10^4$} & \multicolumn{1}{c}{$10^5$} & \multicolumn{1}{c}{$10^6$} & \multicolumn{1}{c}{$10^4$} & \multicolumn{1}{c}{$10^5$} & \multicolumn{1}{c}{$10^6$} & \multicolumn{1}{c}{$10^4$} & \multicolumn{1}{c}{$10^5$} & \multicolumn{1}{c}{$10^6$} & \multicolumn{1}{c}{$10^4$} & \multicolumn{1}{c}{$10^5$} & \multicolumn{1}{c}{$10^6$} \\
        \midrule 
q        Time [s] & \textsc{SelectionSelect} & 0.05569 & 5.51715 & - & 0.05480 & 5.49899 & - & 0.05839 & 6.95391 & - & 0.05491 & 5.51944 & - \\
        & \textsc{HeapSelect} & 0.00078 & 0.00940 & 0.11738 & 0.00059 & 0.00700 & 0.08255 & 0.00067 & 0.00755 & 0.09756 & 0.00005 & 0.00051 & 0.00551 \\
        & \textsc{QuickSelect} & 0.00014 & 0.00140 & - & 0.12229 & 12.39652 & - & 0.11824 & 11.03308 & - & 0.12279 & 12.37940 & - \\
        & \textsc{FRSelect} & 0.00010 & 0.00083 & 0.00717 & 0.00001 & 0.00015 & 0.00223 & 0.00010 & 0.00081 & 0.00744 & 0.00005 & 0.00045 & 0.00465 \\
        \midrule 
        \#comps & \textsc{SelectionSelect} & 37502499 & 3750024999 & - & 37502499 & 3750024999 & - & 37502499 & 3750024999 & - & 37502499 & 3750024999 & - \\
        & \textsc{HeapSelect} & 137082 & 1704351 & 20315900 & 128095 & 1609845 & 19421172 & 142375 & 1756943 & 20865881 & 20001 & 200001 & 2000001 \\
        & \textsc{QuickSelect} & 33510 & 337053 & - & 37497500 & 3749975000 & - & 49530691 & 4567259156 & - & 37497500 & 3749975000 & - \\
        & \textsc{FRSelect} & 28662 & 228256 & 2053122 & 10237 & 128132 & 3023096 & 60642 & 516838 & 3715446 & 21036 & 205538 & 2027434 \\
        \bottomrule
    \end{tabular}
\end{table}

\begin{table}[h!]
    \centering
    \scriptsize
    \caption{Computation times and the number of computations of the algorithms for the search of the $10$-th percentile ($k=N/10$).}
    \label{table:k-0.1N}
    \begin{tabular}{llrrrrrrrrrrrr}
        \toprule
        {Metric} & {Type of array} & \multicolumn{3}{c}{{random}} & \multicolumn{3}{c}{{increasing}} & \multicolumn{3}{c}{{decreasing}} & \multicolumn{3}{c}{{constant}} \\
        \cmidrule(r){3-5} \cmidrule(lr){6-8} \cmidrule(lr){9-11} \cmidrule(l){12-14}
        & {Size} & \multicolumn{1}{c}{$10^4$} & \multicolumn{1}{c}{$10^5$} & \multicolumn{1}{c}{$10^6$} & \multicolumn{1}{c}{$10^4$} & \multicolumn{1}{c}{$10^5$} & \multicolumn{1}{c}{$10^6$} & \multicolumn{1}{c}{$10^4$} & \multicolumn{1}{c}{$10^5$} & \multicolumn{1}{c}{$10^6$} & \multicolumn{1}{c}{$10^4$} & \multicolumn{1}{c}{$10^5$} & \multicolumn{1}{c}{$10^6$} \\
        \midrule 
        Time [s] & \textsc{SelectionSelect} & 0.01400 & 1.40042 & - & 0.01396 & 1.39636 & - & 0.01574 & 1.94957 & - & 0.01389 & 1.39717 & - \\
        & \textsc{HeapSelect} & 0.00023 & 0.00282 & 0.03382 & 0.00013 & 0.00156 & 0.01867 & 0.00018 & 0.00215 & 0.02580 & 0.00003 & 0.00028 & 0.00290 \\
        & \textsc{QuickSelect} & 0.00011 & 0.00124 & - & 0.16220 & 16.31266 & - & 0.04240 & 3.97832 & - & 0.16169 & 16.34995 & - \\
        & \textsc{FRSelect} & 0.00006 & 0.00043 & 0.00359 & 0.00002 & 0.00016 & 0.00208 & 0.00006 & 0.00054 & 0.00538 & 0.00005 & 0.00046 & 0.00467 \\
        \midrule
        \#comps & \textsc{SelectionSelect} & 9508499 & 950084999 & - & 9508499 & 950084999 & - & 9508499 & 950084999 & - & 9508499 & 950084999 & - \\
        & \textsc{HeapSelect} & 43315 & 499237 & 5645068 & 34385 & 411295 & 4778017 & 45756 & 517608 & 5783012 & 12001 & 120001 & 1200001 \\
        & \textsc{QuickSelect} & 26776 & 292732 & - & 49495500 & 4949955000 & - & 17828014 & 1642308451 & - & 49495500 & 4949955000 & - \\
        & \textsc{FRSelect} & 16991 & 148697 & 1538045 & 10238 & 110162 & 1634026 & 42912 & 380860 & 3564183 & 21144 & 205874 & 2028430 \\
        \bottomrule
    \end{tabular}
\end{table}


\subsection{Comments} \label{comment}

\subsubsection{Compare the evolution of computation times} \label{sec:comment-a}

Table \ref{table:evolution-observation} shows the increasing rates of computation times from $N=10^4$ to $N=10^5$ and from $N=10^5$ to $N=10^6$ regarding the experiments, while Table \ref{table:evolution-theory} provides the expected growth based on the complexity. 

\begin{table}[h!]
    \centering
    \footnotesize
    \caption{The observed evolution of computation times.}
    \label{table:evolution-observation}
    \begin{tabular}{llrrrrrrrrr}
        \toprule
        Evolution range & {Type of array} & \multicolumn{2}{c}{{random}} & \multicolumn{2}{c}{{increasing}} & \multicolumn{2}{c}{{decreasing}} & \multicolumn{2}{c}{{constant}} \\
        \cmidrule(r){3-4} \cmidrule(lr){5-6} \cmidrule(lr){7-8} \cmidrule(l){9-10}
        & $k$ & $N/2$ & $N/10$ & $N/2$ & $N/10$ & $N/2$ & $N/10$ & $N/2$ & $N/10$ \\
        \midrule
        $10^4 \to 10^5$ & \textsc{SelectionSelect} & 99.07 & 100.03 & 100.35 & 100.03 & 119.09 & 123.86 & 100.52 & 100.59 \\
        & \textsc{HeapSelect} & 12.05 & 12.26 & 11.86 & 12.00 & 11.27 & 11.94 & 10.20 & 9.33 \\
        & \textsc{QuickSelect} & 10.00 & 11.27 & 101.37 & 100.57 & 93.31 & 93.83 & 100.82 & 101.12 \\
        & \textsc{FRSelect} & 8.30 & 7.17 & 15.00 & 8.00 & 8.10 & 9.00 & 9.00 & 9.20 \\
             \midrule
        $10^5 \to 10^6$ & \textsc{SelectionSelect} & - & - & - & - & - & - & - & - \\
        & \textsc{HeapSelect} & 12.49 & 11.99 & 11.79 & 11.97 & 12.92 & 12.00 & 10.80 & 10.36 \\
        & \textsc{QuickSelect} & - & - & - & - & - & - & - & - \\
        & \textsc{FRSelect} & 8.64 & 8.35 & 14.87 & 13.00 & 9.19 & 9.96 & 10.33 & 10.15 \\      
        \bottomrule
    \end{tabular}
\end{table}

\begin{table}[h!]
    \centering
    \footnotesize
    \caption{The theoretical evolution in terms of complexity.}
    \label{table:evolution-theory}
    \begin{tabular}{lrrrrrrrrr}
        \toprule
        {Type of complexity} & \multicolumn{2}{c}{{$\Theta (Nk)$}} & \multicolumn{2}{c}{{$\Theta (N + k \log N)$}} & \multicolumn{2}{c}{{$\Theta (N)$}} & \multicolumn{2}{c}{{$\Theta (N^2)$}} \\
        \cmidrule(r){2-3} \cmidrule(lr){4-5} \cmidrule(lr){6-7} \cmidrule(l){8-9}
        $k$ & $N/2$ & $N/10$ & $N/2$ & $N/10$ & $N/2$ & $N/10$ & $N/2$ & $N/10$ \\
        \midrule
        $10^4 \to 10^5$ & 100.00 & 100.00 & 12.17 & 11.43 & 10.00 & 10.00 & 100.00 & 100.00 \\
        $10^5 \to 10^6$ & 100.00 & 100.00 & 11.79 & 11.25 & 10.00 & 10.00 & 100.00 & 100.00 \\
        \bottomrule
    \end{tabular}
\end{table}

According to Table \ref{table:evolution-observation} and Table \ref{table:evolution-theory} as well as Table \ref{tab:complexity}, the experiments generally match these expectations:

\begin{itemize}
    \item \texttt{SelectionSelect} consistently witnesses a near $100$-time increase, regardless of array type or $k$, which aligns perfectly with its $\Theta(Nk)$ time complexity. Since both $N$ and $k$ grow by $10$ times, the overall work scales by $100$ times.
    \item \texttt{HeapSelect} demonstrates a time increase in the range of $9.33$ to $12.26$ times, closely matching the theoretical complexity of $\Theta (N + k \log N)$. The growth is slightly sublinear in $k$ as expected (both $N$ and $k$ linearly grow by $10$ times, while $\log N$ increases slightly over $1$).
    \item \texttt{QuickSelect} aligns well with its average-case complexity $\Theta (N)$ only for random arrays, where pivot positions are likely to be balanced. However, for sorted and constant arrays, it exhibits $\Theta (N^2)$ behavior, due to always choosing the last element as pivot. This leads to highly unbalanced partitions, which aligns to the worst case of complexity provided in Section \ref{sec:theory-complexity-results}, and thus a recursive depth approaching $N$, as confirmed by over $100$-time growth in time.
    \item \texttt{FRSelect} shows the most consistent and optimal growth, close to 8-10 times, which aligns with its theoretical $\Theta (N)$ performance across all array types. It performance is stable even on sorted, constant arrays, unlike \texttt{QuickSelect}.
\end{itemize}
The results of \texttt{SelectionSelect} and \texttt{QuickSelect} in terms of $N=10^6$ are expected to correspond to the increase of complexity from $N=10^5$ to $N=10^6$.

\subsubsection{Comment on the relative order of the different algorithms}


Based on the experiment results, the relative order depends on the type of array:
\begin{itemize}
    \item For random arrays: \texttt{FRSelect} < \texttt{QuickSelect} < \texttt{HeapSelect} < \texttt{SelectionSelect}
    \item For increase, decrease or constant arrays: \texttt{FRSelect} < \texttt{HeapSelect} < \texttt{SelectionSelect} < \texttt{QuickSelect}
\end{itemize}

In general, \texttt{FRSelect} dominates when input order is random or when values are uniform, while \texttt{HeapSelect} is the best fallback in worst-aligned cases. \texttt{QuickSelect} is highly efficient only under favorable pivot conditions.
\begin{itemize}
    \item \texttt{FRSelect} consistently outperforms the other algorithms across all array types. Its refined pivot selection, ensures balanced partitioning and avoids worst-case behavior. It achieves linear performance in all tested cases, closely matching its theoretical complexity of $\Theta(N)$.
    \item \texttt{QuickSelect} performs very well on random arrays due to its average-case complexity of $\Theta(N)$. However, it performs poorly on sorted or constant arrays because the current implementation always chooses the last element as pivot, resulting in highly unbalanced partitions and $\Theta(N^2)$ behavior in the worst case. This caused significant slowdowns and made it infeasible to run for large inputs like $N = 10^6$.
    \item \texttt{HeapSelect} exhibits stable and predictable performance, regardless of the input type. Its complexity of $\Theta(N + k \log N)$ reflects the cost of heap construction and repeated extract-min operations. While not the fastest, it avoids the extreme behavior seen in QuickSelect and consistently ranks in the middle of the group.
    \item \texttt{SelectionSelect} is the slowest algorithm in all caeses. Its simply design resembles selection sort and has a time complexity of $\Theta(Nk)$, making it scale poorly with both $n$ and $k$. As a result, it becomes impractical for large datasets and was unable to complete execution for $N = 10^6$ due to resource limitations.
\end{itemize}







\subsubsection{Discuss the impact of the parameter $k$}

Table \ref{table:k-impact} presents the ratio of computation times between selecting the median ($k=N/2$) and the 10th percentile ($k=N/10$) for each algorithm.

\begin{table}[h!]
    \centering
    \footnotesize
    \caption{Ratio of computation times between selecting the median and the $10$-th percentile.}
    \label{table:k-impact}
    \begin{tabular}{llrrrrrrrrrrrr}
        \toprule
        {Type of array} & \multicolumn{3}{c}{{random}} & \multicolumn{3}{c}{{increasing}} & \multicolumn{3}{c}{{decreasing}} & \multicolumn{3}{c}{{constant}} \\
        \cmidrule(r){2-4} \cmidrule(lr){5-7} \cmidrule(lr){8-10} \cmidrule(l){11-13}
        {Size} & \multicolumn{1}{c}{$10^4$} & \multicolumn{1}{c}{$10^5$} & \multicolumn{1}{c}{$10^6$} & \multicolumn{1}{c}{$10^4$} & \multicolumn{1}{c}{$10^5$} & \multicolumn{1}{c}{$10^6$} & \multicolumn{1}{c}{$10^4$} & \multicolumn{1}{c}{$10^5$} & \multicolumn{1}{c}{$10^6$} & \multicolumn{1}{c}{$10^4$} & \multicolumn{1}{c}{$10^5$} & \multicolumn{1}{c}{$10^6$} \\
        \midrule 
        SelectionSelect & 3.98 & 3.94 & - & 3.93 & 3.94 & - & 3.71 & 3.57 & - & 3.95 & 3.95 & - \\
        HeapSelect & 3.39 & 3.33 & 3.47 & 4.54 & 4.49 & 4.42 & 3.72 & 3.51 & 3.78 & 1.67 & 1.82 & 1.90 \\
        QuickSelect & 1.27 & 1.13 & - & 0.75 & 0.76 & - & 2.79 & 2.77 & - & 0.76 & 0.76 & - \\
        FRSelect & 1.67 & 1.93 & 2.00 & 0.50 & 0.94 & 1.07 & 1.67 & 1.50 & 1.38 & 1.00 & 0.98 & 1.00 \\
        \bottomrule
    \end{tabular}
\end{table}

Overall, only \texttt{SelectionSelect} and \texttt{HeapSelect} exhibit strong dependences on $k$, while \texttt{QuickSelect} and \texttt{FRSelect} tend to behave uniformly regardless of $k$, assuming favorable pivot distributions:



\begin{itemize}
    \item \texttt{SelectionSelect} is directly and significantly influenced by $k$, as it performs one full scan per increment of $i$ up to $k$. As a result, time grows linearly with $k$, as confirmed by the 4-time difference between times for $k = N/10$ and $k = N/2$.
    \begin{itemize}
        \item The 4-time difference is reasonable. Although $k$ increases by a factor of $5$ between the median and $10$-th percentile due to the time complexity of $\Theta (Nk)$, and the inner loop becomes slightly shorter for higher values of $j$ in the outer loop. As a result, the growth in runtime is sublinear in practice despite the theoretical $\Theta (Nk)$ bound, and an around $4$-time increase instead of $5$-time aligns well with this behavior.
    \end{itemize}
    \item \texttt{HeapSelect} is also impacted by $k$, but less dramatically. Since each \textit{MinHeapify} operation costs logarithmic time, performance scales in the order of $k \log N$, which was visible in the time increases observed.
    \item \texttt{QuickSelect} and \texttt{FRSelect} were generally insensitive to the value of $k$. In these algorithms, the number of recursive steps depends on the pivot position rather than $k$ itself. However, input structure still plays a larger role than $k$ in determining actual performance.
\end{itemize}

\newpage

\bibliographystyle{plain}
\bibliography{references}

\appendix

\section{Appendix: Pseudocode} \label{Appendix}
\begin{multicols}{2}
\subsection{\texttt{SelectionSort}}
\begin{codebox}
\Procname{$\proc{SelectionSelect}(A, k)$}
\li \For $i \gets 1$ \To $k$
\li \Do $smallest \gets i$
\li     \For $j \gets i + 1$ \To $A.length$
\li     \Do \If $A[j] < A[smallest]$
\li         \Then $smallest \gets j$
            \End
        \End
\li     $swap(A[i], A[smallest])$
    \End
\li \Return $k$
\End
\end{codebox}
% \vspace{5pt}

\subsection{\texttt{HeapSelect}}

\begin{codebox}
\Procname{$\proc{HeapSelect}(A, k)$}
\li $\proc{BuildMinHeap}(A)$
\li $A.heap\_size \gets A.length$
\li \For $i \gets 1$ \To $k$
\li \Do $swap(A[1], A[A.heapSize - 1])$
\li     $heap\_size \gets heap\_size - 1$
\li     $\proc{MinHeapify}(A, 1)$
    \End
\li \Return $A.length - k$ // or heap\_size
\End
\end{codebox}

% \vspace{5pt}

\begin{codebox}
\Procname{$\proc{BuildMinHeap}(A)$}
\li \For $i \gets \lfloor length/2 \rfloor$ \Downto $1$
\li \Do $\proc{MinHeapify}(A, i)$
    \End
\End
\end{codebox}

% \vspace{5pt}

% \columnbreak


\begin{codebox}
\Procname{$\proc{minHeapify}(A, i)$}
\li $smallest \gets i$
\li $l \gets LEFT(i)$ // 2i
\li $r \gets RIGHT(i)$ // 2i + 1
\li \If $l \leq heap\_size$ and $A[l] < A[smallest]$
\li \Then $smallest \gets l$
    \End
\li \If $r \leq heap\_size$ and $A[r] < A[smallest]$
\li \Then $smallest \gets r$
    \End
\li \If $smallest \neq i$
\li \Then $swap(A[i], A[smallest])$
\li     $\proc{minHeapify}(A, smallest)$
    \End
\End
\end{codebox}


\subsection{\texttt{QuickSelect}}

\begin{codebox}
\Procname{$\proc{QuickSelect}(A, p, r, k)$}
\li \If $p = r$
\li \Then \Return $p$
    \End
\li $q \gets \proc{Partition}(A, p, r)$
\li \If $k = q$
\li \Then \Return $q$
    \End
\li \ElseIf $k < q$
\li \Then \Return $\proc{QuickSelect}(A, p, q-1, k)$
\li \Else \Return $\proc{QuickSelect}(A, q+1, r, k)$
    \End
\end{codebox}

% \vspace{5pt}

\begin{codebox}
\Procname{$\proc{Partition}(A, p, r)$}
% \li $pivotIndex \gets r$
\li $i \gets p - 1$
\li \For $j \gets p$ \To $r - 1$
\li \Do \If $A[j] \leq A[r]$
\li     \Then $i \gets i + 1$
\li         $swap(A[i], A[j])$
        \End
    \End
\li $swap(A[i+1], A[r])$
\li \Return $i + 1$
\End
\end{codebox}

\subsection{\texttt{FRSelect}}

Floyd et al. introduced the Floyd-Rivest Algorithm with the pseudocode \cite{floyd1975algorithm}.

\end{multicols}

\end{document}

% %% QUESTIONS
% 0/ Best case means the array is sorted or k = 1?
% 1/ HeapSort: Can I initialize `largest = i` first in the MaxHeapify function
% 2/ HeapSelect: floor() function

% 6/ QuickSelect is slower than SelectionSelect and HeapSelect

% 10/ Connect to test machine

%% REFERENCES

% SelectionSelect: https://www.geeksforgeeks.org/selection-sort-algorithm-2/ and https://www.geeksforgeeks.org/selection-algorithms/
% HeapSelect: https://medium.com/nerd-for-tech/quick-select-algorithm-17ac146b6218#:~:text=The%20time%20complexity%20for%20the,be%20avoided%20in%20most%20cases.
% https://malyasova.github.io/data_structures/2020/09/06/selection-problem-heap.html
% QuickSelect: https://www.geeksforgeeks.org/quickselect-algorithm/
% FRSelect: https://www.geeksforgeeks.org/floyd-rivest-algorithm/






% duvu@duvu-Legion-Pro-5-16IRX8:~/All/82_Master_Uliege/INFO0902-SDA/10_Project/project1$ ./heapselect 
% Running experiments on array size: 10000
% Each result is averaged over 10 runs.

% Results for Median (k = 5000)
% Array         Time [s]   (#swaps, #comps)
% ------------  --------   ----------------------
% random         0.00090   (  70743,  137082)
% increasing     0.00058   (  63077,  128095)
% decreasing     0.00066   (  75883,  142375)
% constant       0.00005   (   5001,   20001)

% Results for 10th Percentile (k = 1000)
% Array         Time [s]   (#swaps, #comps)
% ------------  --------   ----------------------
% random         0.00024   (  20463,   43315)
% increasing     0.00014   (  13194,   34385)
% decreasing     0.00018   (  23782,   45756)
% constant       0.00003   (   1001,   12001)

% Results for Median (k = 50000)
% Array         Time [s]   (#swaps, #comps)
% ------------  --------   ----------------------
% random         0.00977   ( 874130, 1704351)
% increasing     0.00731   ( 796792, 1609845)
% decreasing     0.00802   ( 925799, 1756943)
% constant       0.00050   (  50001,  200001)

% Results for 10th Percentile (k = 10000)
% Array         Time [s]   (#swaps, #comps)
% ------------  --------   ----------------------
% random         0.00296   ( 237885,  499237)
% increasing     0.00165   ( 165649,  411295)
% decreasing     0.00223   ( 268201,  517608)
% constant       0.00026   (  10001,  120001)

% duvu@duvu-Legion-Pro-5-16IRX8:~/All/82_Master_Uliege/INFO0902-SDA/10_Project/project1$ ./selectionselect 
% Running experiments on array size: 10000
% Each result is averaged over 10 runs.

% Results for Median (k = 5000)
% Array         Time [s]   (#swaps, #comps)
% ------------  --------   ----------------------
% random         0.05676   (   5001, 37502499)
% increasing     0.05500   (   5001, 37502499)
% decreasing     0.05872   (   5001, 37502499)
% constant       0.05505   (   5001, 37502499)

% Results for 10th Percentile (k = 1000)
% Array         Time [s]   (#swaps, #comps)
% ------------  --------   ----------------------
% random         0.01384   (   1001, 9508499)
% increasing     0.01401   (   1001, 9508499)
% decreasing     0.01519   (   1001, 9508499)
% constant       0.01390   (   1001, 9508499)

% Results for Median (k = 50000)
% Array         Time [s]   (#swaps, #comps)
% ------------  --------   ----------------------
% random         5.54020   (  50001, 3750024999)
% increasing     5.51829   (  50001, 3750024999)
% decreasing     6.96218   (  50001, 3750024999)
% constant       5.49540   (  50001, 3750024999)

% Results for 10th Percentile (k = 10000)
% Array         Time [s]   (#swaps, #comps)
% ------------  --------   ----------------------
% random         1.40004   (  10001, 950084999)
% increasing     1.39025   (  10001, 950084999)
% decreasing     1.94467   (  10001, 950084999)
% constant       1.39178   (  10001, 950084999)

% duvu@duvu-Legion-Pro-5-16IRX8:~/All/82_Master_Uliege/INFO0902-SDA/10_Project/project1$ ./quickselect 
% Running experiments on array size: 10000
% Each result is averaged over 10 runs.

% Results for Median (k = 5000)
% Array         Time [s]   (#swaps, #comps)
% ------------  --------   ----------------------
% random         0.00014   (  16261,   33510)
% increasing     0.12229   (37502500, 37497500)
% decreasing     0.11824   (24749112, 49530691)
% constant       0.12279   (37502500, 37497500)

% Results for 10th Percentile (k = 1000)
% Array         Time [s]   (#swaps, #comps)
% ------------  --------   ----------------------
% random         0.00011   (  14136,   26776)
% increasing     0.16220   (49504500, 49495500)
% decreasing     0.04240   (8913100, 17828014)
% constant       0.16169   (49504500, 49495500)

% Results for Median (k = 50000)
% Array         Time [s]   (#swaps, #comps)
% ------------  --------   ----------------------
% random         0.00140   ( 184300,  337053)
% increasing    12.39652   (3750025000, 3749975000)
% decreasing    11.03308   (2290664101, 4567259156)
% constant      12.37940   (3750025000, 3749975000)

% Results for 10th Percentile (k = 10000)
% Array         Time [s]   (#swaps, #comps)
% ------------  --------   ----------------------
% random         0.00124   ( 119222,  292732)
% increasing    16.31266   (4950045000, 4949955000)
% decreasing     3.97832   (824475759, 1642308451)
% constant      16.34995   (4950045000, 4949955000)